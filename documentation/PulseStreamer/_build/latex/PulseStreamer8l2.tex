% Generated by Sphinx.
\def\sphinxdocclass{report}
\documentclass[letterpaper,10pt,english]{sphinxmanual}
\usepackage[utf8]{inputenc}
\DeclareUnicodeCharacter{00A0}{\nobreakspace}
\usepackage{cmap}
\usepackage[T1]{fontenc}
\usepackage{babel}
\usepackage{times}
\usepackage[Bjarne]{fncychap}
\usepackage{longtable}
\usepackage{sphinx}
\usepackage{multirow}


\title{Pulse Streamer 8/2 Documentation}
\date{February 23, 2016}
\release{0.1}
\author{Swabian Instruments}
\newcommand{\sphinxlogo}{}
\renewcommand{\releasename}{Release}
\makeindex

\makeatletter
\def\PYG@reset{\let\PYG@it=\relax \let\PYG@bf=\relax%
    \let\PYG@ul=\relax \let\PYG@tc=\relax%
    \let\PYG@bc=\relax \let\PYG@ff=\relax}
\def\PYG@tok#1{\csname PYG@tok@#1\endcsname}
\def\PYG@toks#1+{\ifx\relax#1\empty\else%
    \PYG@tok{#1}\expandafter\PYG@toks\fi}
\def\PYG@do#1{\PYG@bc{\PYG@tc{\PYG@ul{%
    \PYG@it{\PYG@bf{\PYG@ff{#1}}}}}}}
\def\PYG#1#2{\PYG@reset\PYG@toks#1+\relax+\PYG@do{#2}}

\expandafter\def\csname PYG@tok@gd\endcsname{\def\PYG@tc##1{\textcolor[rgb]{0.63,0.00,0.00}{##1}}}
\expandafter\def\csname PYG@tok@gu\endcsname{\let\PYG@bf=\textbf\def\PYG@tc##1{\textcolor[rgb]{0.50,0.00,0.50}{##1}}}
\expandafter\def\csname PYG@tok@gt\endcsname{\def\PYG@tc##1{\textcolor[rgb]{0.00,0.27,0.87}{##1}}}
\expandafter\def\csname PYG@tok@gs\endcsname{\let\PYG@bf=\textbf}
\expandafter\def\csname PYG@tok@gr\endcsname{\def\PYG@tc##1{\textcolor[rgb]{1.00,0.00,0.00}{##1}}}
\expandafter\def\csname PYG@tok@cm\endcsname{\let\PYG@it=\textit\def\PYG@tc##1{\textcolor[rgb]{0.25,0.50,0.56}{##1}}}
\expandafter\def\csname PYG@tok@vg\endcsname{\def\PYG@tc##1{\textcolor[rgb]{0.73,0.38,0.84}{##1}}}
\expandafter\def\csname PYG@tok@m\endcsname{\def\PYG@tc##1{\textcolor[rgb]{0.13,0.50,0.31}{##1}}}
\expandafter\def\csname PYG@tok@mh\endcsname{\def\PYG@tc##1{\textcolor[rgb]{0.13,0.50,0.31}{##1}}}
\expandafter\def\csname PYG@tok@cs\endcsname{\def\PYG@tc##1{\textcolor[rgb]{0.25,0.50,0.56}{##1}}\def\PYG@bc##1{\setlength{\fboxsep}{0pt}\colorbox[rgb]{1.00,0.94,0.94}{\strut ##1}}}
\expandafter\def\csname PYG@tok@ge\endcsname{\let\PYG@it=\textit}
\expandafter\def\csname PYG@tok@vc\endcsname{\def\PYG@tc##1{\textcolor[rgb]{0.73,0.38,0.84}{##1}}}
\expandafter\def\csname PYG@tok@il\endcsname{\def\PYG@tc##1{\textcolor[rgb]{0.13,0.50,0.31}{##1}}}
\expandafter\def\csname PYG@tok@go\endcsname{\def\PYG@tc##1{\textcolor[rgb]{0.20,0.20,0.20}{##1}}}
\expandafter\def\csname PYG@tok@cp\endcsname{\def\PYG@tc##1{\textcolor[rgb]{0.00,0.44,0.13}{##1}}}
\expandafter\def\csname PYG@tok@gi\endcsname{\def\PYG@tc##1{\textcolor[rgb]{0.00,0.63,0.00}{##1}}}
\expandafter\def\csname PYG@tok@gh\endcsname{\let\PYG@bf=\textbf\def\PYG@tc##1{\textcolor[rgb]{0.00,0.00,0.50}{##1}}}
\expandafter\def\csname PYG@tok@ni\endcsname{\let\PYG@bf=\textbf\def\PYG@tc##1{\textcolor[rgb]{0.84,0.33,0.22}{##1}}}
\expandafter\def\csname PYG@tok@nl\endcsname{\let\PYG@bf=\textbf\def\PYG@tc##1{\textcolor[rgb]{0.00,0.13,0.44}{##1}}}
\expandafter\def\csname PYG@tok@nn\endcsname{\let\PYG@bf=\textbf\def\PYG@tc##1{\textcolor[rgb]{0.05,0.52,0.71}{##1}}}
\expandafter\def\csname PYG@tok@no\endcsname{\def\PYG@tc##1{\textcolor[rgb]{0.38,0.68,0.84}{##1}}}
\expandafter\def\csname PYG@tok@na\endcsname{\def\PYG@tc##1{\textcolor[rgb]{0.25,0.44,0.63}{##1}}}
\expandafter\def\csname PYG@tok@nb\endcsname{\def\PYG@tc##1{\textcolor[rgb]{0.00,0.44,0.13}{##1}}}
\expandafter\def\csname PYG@tok@nc\endcsname{\let\PYG@bf=\textbf\def\PYG@tc##1{\textcolor[rgb]{0.05,0.52,0.71}{##1}}}
\expandafter\def\csname PYG@tok@nd\endcsname{\let\PYG@bf=\textbf\def\PYG@tc##1{\textcolor[rgb]{0.33,0.33,0.33}{##1}}}
\expandafter\def\csname PYG@tok@ne\endcsname{\def\PYG@tc##1{\textcolor[rgb]{0.00,0.44,0.13}{##1}}}
\expandafter\def\csname PYG@tok@nf\endcsname{\def\PYG@tc##1{\textcolor[rgb]{0.02,0.16,0.49}{##1}}}
\expandafter\def\csname PYG@tok@si\endcsname{\let\PYG@it=\textit\def\PYG@tc##1{\textcolor[rgb]{0.44,0.63,0.82}{##1}}}
\expandafter\def\csname PYG@tok@s2\endcsname{\def\PYG@tc##1{\textcolor[rgb]{0.25,0.44,0.63}{##1}}}
\expandafter\def\csname PYG@tok@vi\endcsname{\def\PYG@tc##1{\textcolor[rgb]{0.73,0.38,0.84}{##1}}}
\expandafter\def\csname PYG@tok@nt\endcsname{\let\PYG@bf=\textbf\def\PYG@tc##1{\textcolor[rgb]{0.02,0.16,0.45}{##1}}}
\expandafter\def\csname PYG@tok@nv\endcsname{\def\PYG@tc##1{\textcolor[rgb]{0.73,0.38,0.84}{##1}}}
\expandafter\def\csname PYG@tok@s1\endcsname{\def\PYG@tc##1{\textcolor[rgb]{0.25,0.44,0.63}{##1}}}
\expandafter\def\csname PYG@tok@gp\endcsname{\let\PYG@bf=\textbf\def\PYG@tc##1{\textcolor[rgb]{0.78,0.36,0.04}{##1}}}
\expandafter\def\csname PYG@tok@sh\endcsname{\def\PYG@tc##1{\textcolor[rgb]{0.25,0.44,0.63}{##1}}}
\expandafter\def\csname PYG@tok@ow\endcsname{\let\PYG@bf=\textbf\def\PYG@tc##1{\textcolor[rgb]{0.00,0.44,0.13}{##1}}}
\expandafter\def\csname PYG@tok@sx\endcsname{\def\PYG@tc##1{\textcolor[rgb]{0.78,0.36,0.04}{##1}}}
\expandafter\def\csname PYG@tok@bp\endcsname{\def\PYG@tc##1{\textcolor[rgb]{0.00,0.44,0.13}{##1}}}
\expandafter\def\csname PYG@tok@c1\endcsname{\let\PYG@it=\textit\def\PYG@tc##1{\textcolor[rgb]{0.25,0.50,0.56}{##1}}}
\expandafter\def\csname PYG@tok@kc\endcsname{\let\PYG@bf=\textbf\def\PYG@tc##1{\textcolor[rgb]{0.00,0.44,0.13}{##1}}}
\expandafter\def\csname PYG@tok@c\endcsname{\let\PYG@it=\textit\def\PYG@tc##1{\textcolor[rgb]{0.25,0.50,0.56}{##1}}}
\expandafter\def\csname PYG@tok@mf\endcsname{\def\PYG@tc##1{\textcolor[rgb]{0.13,0.50,0.31}{##1}}}
\expandafter\def\csname PYG@tok@err\endcsname{\def\PYG@bc##1{\setlength{\fboxsep}{0pt}\fcolorbox[rgb]{1.00,0.00,0.00}{1,1,1}{\strut ##1}}}
\expandafter\def\csname PYG@tok@kd\endcsname{\let\PYG@bf=\textbf\def\PYG@tc##1{\textcolor[rgb]{0.00,0.44,0.13}{##1}}}
\expandafter\def\csname PYG@tok@ss\endcsname{\def\PYG@tc##1{\textcolor[rgb]{0.32,0.47,0.09}{##1}}}
\expandafter\def\csname PYG@tok@sr\endcsname{\def\PYG@tc##1{\textcolor[rgb]{0.14,0.33,0.53}{##1}}}
\expandafter\def\csname PYG@tok@mo\endcsname{\def\PYG@tc##1{\textcolor[rgb]{0.13,0.50,0.31}{##1}}}
\expandafter\def\csname PYG@tok@mi\endcsname{\def\PYG@tc##1{\textcolor[rgb]{0.13,0.50,0.31}{##1}}}
\expandafter\def\csname PYG@tok@kn\endcsname{\let\PYG@bf=\textbf\def\PYG@tc##1{\textcolor[rgb]{0.00,0.44,0.13}{##1}}}
\expandafter\def\csname PYG@tok@o\endcsname{\def\PYG@tc##1{\textcolor[rgb]{0.40,0.40,0.40}{##1}}}
\expandafter\def\csname PYG@tok@kr\endcsname{\let\PYG@bf=\textbf\def\PYG@tc##1{\textcolor[rgb]{0.00,0.44,0.13}{##1}}}
\expandafter\def\csname PYG@tok@s\endcsname{\def\PYG@tc##1{\textcolor[rgb]{0.25,0.44,0.63}{##1}}}
\expandafter\def\csname PYG@tok@kp\endcsname{\def\PYG@tc##1{\textcolor[rgb]{0.00,0.44,0.13}{##1}}}
\expandafter\def\csname PYG@tok@w\endcsname{\def\PYG@tc##1{\textcolor[rgb]{0.73,0.73,0.73}{##1}}}
\expandafter\def\csname PYG@tok@kt\endcsname{\def\PYG@tc##1{\textcolor[rgb]{0.56,0.13,0.00}{##1}}}
\expandafter\def\csname PYG@tok@sc\endcsname{\def\PYG@tc##1{\textcolor[rgb]{0.25,0.44,0.63}{##1}}}
\expandafter\def\csname PYG@tok@sb\endcsname{\def\PYG@tc##1{\textcolor[rgb]{0.25,0.44,0.63}{##1}}}
\expandafter\def\csname PYG@tok@k\endcsname{\let\PYG@bf=\textbf\def\PYG@tc##1{\textcolor[rgb]{0.00,0.44,0.13}{##1}}}
\expandafter\def\csname PYG@tok@se\endcsname{\let\PYG@bf=\textbf\def\PYG@tc##1{\textcolor[rgb]{0.25,0.44,0.63}{##1}}}
\expandafter\def\csname PYG@tok@sd\endcsname{\let\PYG@it=\textit\def\PYG@tc##1{\textcolor[rgb]{0.25,0.44,0.63}{##1}}}

\def\PYGZbs{\char`\\}
\def\PYGZus{\char`\_}
\def\PYGZob{\char`\{}
\def\PYGZcb{\char`\}}
\def\PYGZca{\char`\^}
\def\PYGZam{\char`\&}
\def\PYGZlt{\char`\<}
\def\PYGZgt{\char`\>}
\def\PYGZsh{\char`\#}
\def\PYGZpc{\char`\%}
\def\PYGZdl{\char`\$}
\def\PYGZhy{\char`\-}
\def\PYGZsq{\char`\'}
\def\PYGZdq{\char`\"}
\def\PYGZti{\char`\~}
% for compatibility with earlier versions
\def\PYGZat{@}
\def\PYGZlb{[}
\def\PYGZrb{]}
\makeatother

\renewcommand\PYGZsq{\textquotesingle}

\begin{document}

\maketitle
\tableofcontents
\phantomsection\label{index::doc}
{\hfill\includegraphics[width=0.900\linewidth]{logo_as_paths.png}\hfill}




\chapter{Hardware}
\label{sections/hardware:hardware}\label{sections/hardware::doc}\label{sections/hardware:welcome-to-pulse-streamer-s-documentation}

\section{Output Channels}
\label{sections/hardware:output-channels}
The \emph{The Pulse Streamer 8/2} has 8 digital and two analog output channels.
The electrical characteristics are tabulated below.


\subsection{Digital Output}
\label{sections/hardware:digital-output}
\begin{tabulary}{\linewidth}{|L|L|}
\hline
\textsf{\relax 
Property
} & \textsf{\relax 
Value
}\\
\hline
Voltage level
 & 
0 to 3.3 V
\\

Output drive
 & 
50 \(\Omega\)
\\

Sampling rate
 & 
1 GHz
\\

Bandwidth
 & 
300 MHz
\\
\hline\end{tabulary}



\subsection{Analog Output}
\label{sections/hardware:analog-output}
\begin{tabulary}{\linewidth}{|L|L|}
\hline
\textsf{\relax 
Property
} & \textsf{\relax 
Value
}\\
\hline
Output Voltage Range
 & 
-1.0 to 1.0 V
\\

Output drive
 & 
50 \(\Omega\)
\\

Sampling rate
 & 
125 MHz
\\

Bandwidth
 & 
80 MHz
\\
\hline\end{tabulary}



\section{Trigger Input}
\label{sections/hardware:trigger-input}
The Pulse Streamer 8/2 has one trigger input that can be applied to GPIO 3.


\section{General purpose IO (available upon request)}
\label{sections/hardware:general-purpose-io-available-upon-request}
The device is ready to be equipped with SMA connectorized general purpose digital IO ports and
slow analog input and output ports. These can be used to implement custom features such as special fast input or
output triggers, enable / disable gates, software controllable input and output lines, etc..
Please contact us for custom designs.


\chapter{Programming Interface}
\label{sections/interface:programming-interface}\label{sections/interface::doc}

\section{Pulse Sequences}
\label{sections/interface:pulse-sequences}
Pulse sequences are represented  as one dimensional arrays of pulses.
Each pulse specifies its duration and the states of the digital
and analog output channels. The C++ data type is:

\begin{Verbatim}[commandchars=\\\{\}]
struct Pulse \PYGZob{}
    unsigned int ticks; // duration in ns
    unsigned char digi; // bit mask
    short ao0;
    short ao1;
\PYGZcb{};
\end{Verbatim}

The pulse duration is specified in nanoseconds.

The lowest bit in the digital bit mask ``digi'' corresponds to channel 0, the highest bit to channel 7.
A channel is high when its corresponding bit is 1 and low otherwise.

The analog values span the full signed 16 bit integer range, i.e. -1.0 V corresponds
to -0x7fff and 1.0 V corresponds to 0x7fff. Note that the DAC resolution is 12 bits,
i.e., the 4 LSB are ignored.


\section{Running pulse sequences}
\label{sections/interface:running-pulse-sequences}
Running a pulse sequence corresponds to a single function call where you
pass your pulse sequence as an argument.

You can repeat a pulse sequence indefinitely or an integer number of times.
In the latter case, you can additionally specify the output state after the
execution of the sequence.

By default, the sequence is executed immediately. Alternatively, you can
tell the system to wait for an external trigger applied to GPIO channel 3.

The C++ method to run a pulse sequence is:

\begin{Verbatim}[commandchars=\\\{\}]
void stream(std::vector\PYGZlt{}Pulse\PYGZgt{} sequence,
            unsigned long n\PYGZus{}runs=0,
            Pulse final=Pulse\PYGZob{}0,0,0,0\PYGZcb{},
            Pulse underflow=Pulse\PYGZob{}0,0,0,0\PYGZcb{},
            bool triggered=False
            )
\end{Verbatim}

sequence represents the pulse sequence

the sequence is repeated indefinitely if n\_runs \textless{}=0 and a finite number of repetitions otherwise.

final represents the final output state after completing the sequence (the tick value is ignored).

underflow specifies the output state that the instrument should enter in case a buffer undeflow occurs (the tick value is ignored).

If trigger is true, the system waits for an external trigger before starting the sequence.


\section{Data streaming and underflows}
\label{sections/interface:data-streaming-and-underflows}
The total sampling rate of the pulse streamer is higher than its internal data transfer rate (\textasciitilde{}6.4 Gbit / s).
Thus, the instrument will run into buffer underflow conditions when the pulse sequence cannot be represented
in a compressed form. This is the case e.g. for very short digital pulses with arbitrary channel masks
or for arbitrary analog sequences with the full sample rate.
However, in most use cases the pulse sequence can be represented in a compressed form, where the
actual rate of transfered information is smaller than the sampling rate.
Buffer underflows will not occurr when the average repeat value of the corresponding low level pulses is
larger or equal to 3.

To treat buffer underflows gracefully, the instrument will halt the output data stream and set the output levels
to a user defined state.

If you are streaming sequences at the edge of the cappability, it is good practice to check for buffer underflows
on a regular basis using the programming interface.


\section{Checking for buffer underflows}
\label{sections/interface:checking-for-buffer-underflows}
You can check for buffer underflows with a single function call. The underlying C++ function is

\begin{Verbatim}[commandchars=\\\{\}]
bool getUnderflow()
\end{Verbatim}

The function returns true when the system has entered the underflow state.


\section{Communicating with the instrument}
\label{sections/interface:communicating-with-the-instrument}
Your Pulse Streamer 8/2 contains an embedded operating system.
You connect to the embedded system over LAN through straight forward
``Remote Procedure Calls'' (RPC). Requests to the system are
directly converted into C++ calls.
This architecture gives you direct control over the system.

You can connect throught two RPC interfaces. (i) a JSON-RPC interface that is based on
the well established JSON data format (\href{http://www.json.org/}{http://www.json.org/})
(ii) a google RPC interface (gRPC) that is based on googles data exchange format
(\href{https://developers.google.com/protocol-buffers/}{https://developers.google.com/protocol-buffers/}).
Both RPC interfaces provide the same functionality.


\section{JSON-RPC Interface}
\label{sections/interface:json-rpc-interface}
JSON-RPC libraries are available for most software languages. More information can be found
on the official website \href{http://json-rpc.org/}{http://json-rpc.org/} and on Wikpedia \href{https://en.wikipedia.org/wiki/JSON-RPC}{https://en.wikipedia.org/wiki/JSON-RPC}.

The JSON-RPC URL of the Pulse Streamer is \href{http://192.168.1.100:8050/json-rpc}{http://192.168.1.100:8050/json-rpc}.


\subsection{Sending Data over JSON-RPC}
\label{sections/interface:sending-data-over-json-rpc}
There is no native format for sending array data over JSON-RPC. Therefore the pulse sequence
is sent as a binary string. Since the http transport layer requires string data to be base64 encoded,
one conversion step is needed before sending a sequence. The JSON-RPC interface call is

\begin{Verbatim}[commandchars=\\\{\}]
\PYGZob{}base64 string sequence,
 unsigned long n\PYGZus{}runs,
 \PYGZob{}unsigned int ticks, unsigned char digi, short ao0, short ao1\PYGZcb{},
 \PYGZob{}unsigned int ticks, unsigned char digi, short ao0, short ao1\PYGZcb{},
 bool triggered\PYGZcb{}
\end{Verbatim}

``sequence'' is the array data as per above C++ data format definition packed into a binary string and converted
to a base64 string. Please check out the python exaple for connecting to the JSON-RPC server
random\_pulses.py.
All other arguments are self explanatory.


\section{gRPC Interface}
\label{sections/interface:grpc-interface}
gRPC (\href{http://www.grpc.io/}{http://www.grpc.io/}) is a new RPC interface that is based on googles well established
data exchange format called Protocol Buffers (\href{https://developers.google.com/protocol-buffers/}{https://developers.google.com/protocol-buffers/}).
There are gRPC libraries available
for most programming languages. Note that gRPC requires the new Protobuf3
standard and is in a beta development stage.

The gRPC server of the Pulse Streamer is 192.168.1.100:50051.


\subsection{Sending Data over gRPC}
\label{sections/interface:sending-data-over-grpc}
In gRPC, data types are defined by generic, language independent templates.
The language specific implementation automatically takes care about conversion
to native data types.

The Pulse Streamer interface looks like this. Please check out the source file
pulse\_streamer.proto.

\begin{Verbatim}[commandchars=\\\{\}]
syntax = \PYGZdq{}proto3\PYGZdq{};
package pulse\PYGZus{}streamer;

message PulseMessage
\PYGZob{}
    uint32 ticks = 1;
    uint32 digi = 2;
    int32 ao0 = 3;
    int32 ao1 = 4;
\PYGZcb{}

message SequenceMessage
\PYGZob{}
    repeated PulseMessage pulse = 1;
    int64 repeat = 2;
    PulseMessage final = 3;
    PulseMessage underflow = 4;
    bool triggered = 5;
\PYGZcb{}

message PulseStreamerReply
\PYGZob{}
    string message = 1;
\PYGZcb{}

service PulseStreamer
\PYGZob{}
    rpc stream (SequenceMessage) returns (PulseStreamerReply) \PYGZob{}\PYGZcb{}
\PYGZcb{}
\end{Verbatim}

Please check out the python example for connecting to the gRPC interface
random\_pulses.py.


\chapter{Indices and tables}
\label{index:indices-and-tables}\begin{itemize}
\item {} 
\emph{genindex}

\item {} 
\emph{modindex}

\item {} 
\emph{search}

\end{itemize}



\renewcommand{\indexname}{Index}
\printindex
\end{document}
